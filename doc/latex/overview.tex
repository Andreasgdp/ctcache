\documentclass[compress,table,xcolor=table]{beamer}
\input{beamer_ctcache.tex}
\setlength{\paperwidth}{17.0cm}
\setlength{\paperheight}{10.0cm}
\setlength{\textwidth}{15.0cm}
\setlength{\textheight}{9.0cm}
\begin{document}
% ------------------------------------------------------------------------------
\title{Speeding up static analysis with \cmdname{clang-tidy-cache}}
% - Intro ----------------------------------------------------------------------
\section{Introduction}
\frame{\titlepage}
% ------------------------------------------------------------------------------
\begin{frame}
  \Huge
  \centering{Introduction}
\end{frame}
% ------------------------------------------------------------------------------
\begin{frame}
  \frametitle{\cmdname{clang-tidy}}
  \begin{itemize}
      \item{\LARGE{\say{a \cmdname{clang}-based C++ static analysis
          tool\footnote{\url{https://clang.llvm.org/extra/clang-tidy/}}}}}
      \begin{itemize}
      \item{an extensible framework for diagnosing and fixing typical
          programming errors,}
      \item{has a comprehensive suite of built-in checks, for}
        \begin{itemize}
        \item{style violations,}
        \item{language misuse,}
        \item{anti-patterns,}
        \item{common bugs,}
        \item{etc.}
        \end{itemize}
      \item{provides a convenient interface for writing new checks,}
      \item{is configurable, with a large set of options.}
      \end{itemize}
  \end{itemize}
\end{frame}
% ------------------------------------------------------------------------------
\begin{frame}
    \frametitle{Using \cmdname{clang-tidy} with \cmdname{cmake}}
    \LARGE
    \begin{itemize}
    \item \cmdname{cmake} has some built-in support for \cmdname{clang-tidy}:
        \Large
        \begin{itemize}
        \item the \inlinecode{CXX\_CLANG\_TIDY} target property\footnote{
                \url{https://cmake.org/cmake/help/latest/manual/cmake-properties.7.html}}.
        \item the generated build system code includes instructions to run
            \cmdname{clang-tidy}, typically chained with the compilation
                commands.
        \end{itemize}
    \end{itemize}
\end{frame}
% ------------------------------------------------------------------------------
\begin{frame}[fragile]
  \Large
  Find the \cmdname{clang-tidy} command:
  \begin{lstlisting}[language=cmake]
  find_program(
    CLANG_TIDY_COMMAND
    clang-tidy
  )
  \end{lstlisting}

  Add executable;
  \begin{lstlisting}[language=cmake]
  add_executable(my_target my_target.cpp)
  \end{lstlisting}

  or library target:
  \begin{lstlisting}[language=cmake]
  add_library(my_target my_target.cpp)
  \end{lstlisting}
\end{frame}
% ------------------------------------------------------------------------------
\begin{frame}[fragile]
  \Large
  If \cmdname{clang-tidy} was found, tell \cmdname{cmake} to check the sources
  as a part of compilation:
  \begin{lstlisting}[language=cmake]
  if(CLANG_TIDY_COMMAND)
    set_target_properties(
      my_target PROPERTIES
      (*@\listinghl{CXX\_CLANG\_TIDY \$\{CLANG\_TIDY\_COMMAND\}}@*)
    )
  endif()
  \end{lstlisting}
\end{frame}
% - Motivation -----------------------------------------------------------------
\section{Motivation}
% ------------------------------------------------------------------------------
\begin{frame}
  \Huge
  \centering{Motivation}
\end{frame}
% ------------------------------------------------------------------------------
\begin{frame}
  \frametitle{The downsides of using \cmdname{clang-tidy}}
    \LARGE
    \begin{itemize}
    \item Running static analysis takes time, {\larger a lot} of time.
        \Large
        \begin{itemize}
        \item Often more time than the actual compilation.
        \item Unacceptable increase in build times.
        \item Switching static analysis on/off:
            \begin{itemize}
            \item Not ideal.
            \item When do we flip the switch?
            \item Typically ends up always off.
            \end{itemize}
        \end{itemize}
    \end{itemize}
\end{frame}
% ------------------------------------------------------------------------------
\begin{frame}[fragile]
  \frametitle{Switching \cmdname{clang-tidy} checks on/off with \cmdname{cmake}}
  \Large
   Add a \cmdname{cmake} option:
  \begin{lstlisting}[language=cmake]
  option(
    (*@\listinghl{WITH\_STATIC\_ANALYSIS}@*)
    "Enable static analysis" ON
  )
  \end{lstlisting}

  Add analysis-related target properties only when switched on:

  \begin{lstlisting}[language=cmake]
  if((*@\listinghl{WITH\_STATIC\_ANALYSIS}@*) and CLANG_TIDY_COMMAND)
    set_target_properties(
      my_target PROPERTIES
      CXX_CLANG_TIDY ${CLANG_TIDY_COMMAND}
    )
  endif()
  \end{lstlisting}

\end{frame}
% ------------------------------------------------------------------------------
\begin{frame}
  \frametitle{The reason for build slowdowns}
    \LARGE
    \begin{itemize}
    \item Typically most of the code that is analysed doesn't change.
        \begin{itemize}
        \Large
        \item It is re-checked with the same result over and over.
        \item Unless you change something in the core sources included
            everywhere.
        \end{itemize}
    \end{itemize}
\end{frame}
% ------------------------------------------------------------------------------
\begin{frame}
  \frametitle{The goal}
    \LARGE
    \begin{itemize}
    \item Have static analysis always on during development.
    \item Don't wait for rechecking of unchanged code.
    \end{itemize}
\end{frame}
% - Solution -------------------------------------------------------------------
\section{Solution}
% ------------------------------------------------------------------------------
\begin{frame}
  \Huge
  \centering{Solution}
\end{frame}
% ------------------------------------------------------------------------------
\begin{frame}
  \frametitle{The solution -- analysis result caching}
    \Large
    \begin{itemize}
    \item If we can uniquely identify a static-analysis tool invocation
        we can store the result and retrieve it when the same invocation
        is repeated.
        \begin{itemize}
        \item Similar to compilation-caching\footnote{
            \url{https://ccache.dev/}}.
        \item Track all inputs of the analysis:
            \begin{itemize}
            \large
            \item configuration options,
            \item command-line arguments,
            \item the source files,
            \item etc.
            \end{itemize}
        \end{itemize}
    \end{itemize}
\end{frame}
% ------------------------------------------------------------------------------
\begin{frame}
    \Huge
    \centering{\texttt{clang-tidy-cache}!}
\end{frame}
% ------------------------------------------------------------------------------
\begin{frame}
  \frametitle{How does it work?}
    \LARGE
    \begin{itemize}
        \item Scans the command-line arguments of \cmdname{clang-tidy} and
            its input:
        \begin{itemize}
            \Large
            \item configuration files,
            \item analysed source files.
        \end{itemize}
        \item Makes a hash uniquely identifying the invocation from
            the above.
    \end{itemize}
\end{frame}
% ------------------------------------------------------------------------------
\begin{frame}
  \frametitle{How does it work? (cont.)}
    \LARGE
    \begin{itemize}
        \item Checks if the hash is in the cache database:
        \begin{itemize}
            \Large
            \item if it is
            \begin{itemize}
            \item don't run \cmdname{clang-tidy} and return immediately,
            \item this is typically {\larger much} faster.
            \end{itemize}
            \item otherwise
            \begin{itemize}
            \item run \cmdname{clang-tidy} and if
                successful\footnote{if there are no warnings or errors reported}
                store the hash.
            \item This means that {\larger sources with findings keep being
                re-checked} and the findings are shown.
            \end{itemize}
        \end{itemize}
    \end{itemize}
\end{frame}
% ------------------------------------------------------------------------------
\begin{frame}[fragile]
  \frametitle{How is it used?}
  \LARGE
  Create\footnote{or use the one in the repository} a wrapper script, like:
  \begin{lstlisting}[language=bash]
  #!/bin/bash
  REAL_CT=/full/path/to/clang-tidy

  /path/to/clang-tidy-cache \
    "${REAL_CT}" "${@}"
  \end{lstlisting}
  Put it into a directory in search \inlinecode{PATH}, before real
  \cmdname{clang-tidy}.
\end{frame}
% ------------------------------------------------------------------------------
\begin{frame}
  \frametitle{Modes of operation}
    \Huge
    \begin{itemize}
        \item Local
        \item Client / Server
    \end{itemize}
\end{frame}
% ------------------------------------------------------------------------------
\begin{frame}
    \frametitle{Local mode}
    \LARGE
    \begin{itemize}
        \item Stores the database in a local directory hierarchy.
        \item Location determined by the \inlinecode{CTCACHE\_DIR} environment
            variable.
        \item By default a sub-tree in the temporary directory.
        \item If you want persistence specify a directory in
            a disk-based file system.
    \end{itemize}
\end{frame}
% ------------------------------------------------------------------------------
\begin{frame}
    \frametitle{Client / server mode}
    \LARGE
    \begin{itemize}
        \item \cmdname{clang-tidy-cache-server}
        \begin{itemize}
            \Large
            \item HTTP server exposing a REST API.
            \item Can be used to store and retrieve hashes from the client.
            \item The client (\cmdname{clang-tide-cache}) can query
                the server -- still {\larger way} faster than running
                \cmdname{clang-tidy}.
        \end{itemize}
    \end{itemize}
\end{frame}
% ------------------------------------------------------------------------------
\begin{frame}[fragile]
  \frametitle{Service -- Rest API}
    \large
    \begin{itemize}
    \item \texttt{http://ctcache:5000/...}
    \begin{itemize}
    \normalsize
    \item \texttt{/cache/<hash>} -- insert \texttt{<hash>} into cache.
    \item \texttt{/is\_cached/<hash>} -- tests if \texttt{<hash>} is cached.
    \item \texttt{/purge\_cache} -- remove all cached hashes.
    \item \texttt{/info} -- static configuration information\footnote{JSON object}.
    \item \texttt{/stats} -- server run-time status information\footnote{JSON object}.
    \item \texttt{/stats/*} -- individual server status readouts\footnote{JSON values}.
    \item \texttt{/stats/ctcache.json} -- long-term persisten server status information\footnote{JSON file}.
    \item \texttt{/images/*} -- status chart images\footnote{SVG}.
    \end{itemize}
    \end{itemize}
\end{frame}
% ------------------------------------------------------------------------------
\begin{frame}
  \frametitle{Server pages}
  \LARGE
  \begin{itemize}
  \item The clang tidy-cache's HTTP server also serves several web pages
    that are designed to be viewed in a browser:
    \begin{itemize}
        \item the {\em dashboard} -- the main one,
        \item SVG plots showing various server statistics.
    \end{itemize}
  \end{itemize}
\end{frame}
% ------------------------------------------------------------------------------
\begin{frame}
  \frametitle{Server dashboard}
  \fitfig{dashboard}
\end{frame}
% - Deployment -----------------------------------------------------------------
\section{Deployment}
% ------------------------------------------------------------------------------
\begin{frame}
  \Huge
  \centering{Deployment}
\end{frame}
% ------------------------------------------------------------------------------
\begin{frame}
  \frametitle{Deploying the server}
  \LARGE
  \begin{itemize}
  \item There are several ways how to run the server:
    \begin{itemize}
    \item just run it in Python,
    \item as a {\em systemd} service,
    \item in a {\em docker} container.
    \end{itemize}
  \end{itemize}
\end{frame}
% ------------------------------------------------------------------------------
\begin{frame}[fragile]
  \frametitle{Using \cmdname{python3}}
  \Large
  If you want to try it out:
  \begin{lstlisting}[language=bash]
  python3 ./clang-tidy-cache-server 
  \end{lstlisting}
  Check command-line arguments:
  \begin{lstlisting}[language=bash]
  python3 ./clang-tidy-cache-server --help
  \end{lstlisting}
  \normalsize
  \begin{verbatim}
  usage: clang-tidy-cache-server
    [-h] [--debug] [--port NUMBER]
    [--save-path FILE-PATH.gz]
    [--save-interval NUMBER]
    [--stats-save-interval NUMBER]
    [--cleanup-interval NUMBER]
    [--stats-path DIR-PATH]
    [--chart-path DIR-PATH]
  \end{verbatim}
\end{frame}
% ------------------------------------------------------------------------------
\begin{frame}[fragile]
  \frametitle{Using \cmdname{systemd} -- installation}
  \Large
  Install server as user service to home directory:
  \begin{lstlisting}[language=bash]
  cd path/to/ctcache_repo
  ./install-user-service
  \end{lstlisting}
  BTW: install client to user's home directory:
  \begin{lstlisting}[language=bash]
  cd path/to/ctcache_repo
  ./install-user-client
  \end{lstlisting}
  Installed files:
  \begin{lstlisting}[language=bash,basicstyle=\footnotesize\ttfamily]
  ~/.local/bin/clang-tidy # default wrapper script
  ~/.local/bin/clang-tidy-cache
  ~/.local/bin/clang-tidy-cache-server
  ~/.local/share/ctcache/static/* # static web files
  ~/.config/ctcache/systemd_env # systemd environment
  ~/.config/systemd/user/ctcache.service
  \end{lstlisting}
\end{frame}
% ------------------------------------------------------------------------------
\begin{frame}[fragile]
  \frametitle{Using \cmdname{systemd} -- service start/stop}
  \LARGE
  Reload the user service files:
  \begin{lstlisting}[language=bash]
  systemctl --user daemon-reload
  \end{lstlisting}
  Start the service:
  \begin{lstlisting}[language=bash]
  systemctl --user start ctcache.service
  \end{lstlisting}
  Stop the service:
  \begin{lstlisting}[language=bash]
  systemctl --user stop ctcache.service
  \end{lstlisting}
\end{frame}
% ------------------------------------------------------------------------------
\begin{frame}[fragile]
  \frametitle{Using \cmdname{systemd}}
  \LARGE
  Permanently enable automatic start of the service:
  \begin{lstlisting}[language=bash]
  systemctl --user enable ctcache.service
  \end{lstlisting}
  Permanently disable automatic start of the service:
  \begin{lstlisting}[language=bash]
  systemctl --user disable ctcache.service
  \end{lstlisting}
\end{frame}
% ------------------------------------------------------------------------------
\begin{frame}[fragile]
  \frametitle{Using \cmdname{docker}}
  \LARGE
  Build the image:
  \begin{lstlisting}[language=bash]
  docker build -t ctcache .
  \end{lstlisting}
  Basic usage:
  \begin{lstlisting}[language=bash]
  docker run \
    -e CTCACHE_PORT=5000 \
    -p "80:5000" \
    -it --rm \
    --name ctcache ctcache
  \end{lstlisting}
\end{frame}
% ------------------------------------------------------------------------------
\begin{frame}[fragile]
  \frametitle{Make cache data persistent with \cmdname{docker} {\em volumes}}
  \LARGE
  Create the volume:
  \begin{lstlisting}[language=bash]
  docker volume create ctcache
  \end{lstlisting}
  Start container using the volume:
  \begin{lstlisting}[language=bash]
  docker run \
    -e CTCACHE_PORT=5000 \
    -p "80:5000" \
    -v "ctcache:/var/lib/ctcache" \
    -it --rm \
    --name ctcache ctcache
  \end{lstlisting}
\end{frame}
% ------------------------------------------------------------------------------
\begin{frame}[fragile]
  \frametitle{Using \cmdname{docker-compose}}
  \LARGE
  The \inlinecode{docker-compose.yaml} file:
  \begin{lstlisting}
  version: "3.6"
  services:
    ctcache:
      build: .
      ports:
        - "5000:5000"
      volumes:
        - "ctcache:/var/lib/ctcache"
  volumes:
    ctcache:
  \end{lstlisting}
\end{frame}
% ------------------------------------------------------------------------------
\begin{frame}[fragile]
  \frametitle{Using \cmdname{docker-compose}}
  \Large
  Start the service container using \cmdname{docker-compose}:
  \begin{lstlisting}[language=bash]
  docker-compose up
  \end{lstlisting}
  as a daemon:
  \begin{lstlisting}[language=bash]
  docker-compose up -d
  \end{lstlisting}
  Stop the running daemon and cleanup the container:
  \begin{lstlisting}[language=bash]
  docker-compose down
  \end{lstlisting}
\end{frame}
% ------------------------------------------------------------------------------
\begin{frame}
  \frametitle{Environment variables}
  \large
  \begin{center}
  {\rowcolors{2}{ctcachepage}{ctcachelisting}
  \begin{tabular}{|p{4cm}|p{1cm}|p{1cm}|p{7cm}|}
  \hline
  variable & client & server & meaning \\
  \hline
      \texttt{CTCACHE\_CLANG\_TIDY} & \checkmark & & path to the \inlinecode{clang-tidy} executable to be used\\
      \texttt{CTCACHE\_DISABLE} & \checkmark & & disables cache, always runs \inlinecode{clang-tidy}\\
      \texttt{CTCACHE\_SKIP} & \checkmark & & disables analysis, client returns \say{OK} immediately \\
      \texttt{CTCACHE\_STRIP} & \checkmark & & list of strings stripped from hashed inputs \\
      \texttt{CTCACHE\_DIR} & \checkmark & & the cache directory in local mode \\
      \texttt{CTCACHE\_HOST} & \checkmark & \checkmark & hostname or IP address of the server \\
      \texttt{CTCACHE\_PORT} & \checkmark & \checkmark & port number on which the server accepts connections \\
      \texttt{CTCACHE\_WEBROOT} & & \checkmark & directory where static served files are located \\
  \hline
  \end{tabular}
  }
  \end{center}
\end{frame}
% - Measurements ---------------------------------------------------------------
\section{Measurements}
% ------------------------------------------------------------------------------
\begin{frame}
  \Huge
  \centering{Measurements}
\end{frame}
% ------------------------------------------------------------------------------
\begin{frame}
  \frametitle{Test projects}
    \Large
    \begin{itemize}
    \item Project 1 (small):
    \begin{itemize}
    \large
    \item proprietary C++ training examples,
    \item some use {\LARGE heavy} template meta-programming,
    \item part of the code is not analyzed,
    \item {\larger $\approx$ 6300 LOC}, {\em \larger tens} of build targets.
    \end{itemize}
    \item Project 2 (medium):
    \begin{itemize}
    \large
    \item open-source C++ wrapper for EGL, OpenGL, OpenAL, \ldots,
    \item \url{https://github.com/matus-chochlik/oglplu2},
    \item {\larger $\approx$ 135k LOC}, {\em \larger hundreds} of build targets.
    \end{itemize}
    \item Project 3 (large):
    \begin{itemize}
    \large
    \item proprietary, production code for embedded HW,
    \item there are some unfixed findings -- some sources are re-checked,
    \item {\larger $\approx$ 750k LOC}, {\em \larger thousands} of build targets.
    \end{itemize}
    \end{itemize}
\end{frame}
% ------------------------------------------------------------------------------
\begin{frame}
  \frametitle{Test hardware setup}
    \Large
    \begin{itemize}
    \item development laptop:
    \begin{itemize}
    \large
    \item i5-7200U @ 2.50GHz (4 cores),
    \item 16GB RAM,
    \item 250GB SSD.
    \end{itemize}
    \item ctcache server:
    \begin{itemize}
    \large
    \item RPi 3B,
    \item BCM2837 64bit @ 1.20GHz (4 cores),
    \item 1GB RAM,
    \item 1TB USB HDD.
    \end{itemize}
    \item connected over WiFi (5GHz / $\approx$ 650Mbps\footnote{
            according to \inlinecode{iwconfig}}).
    \end{itemize}
\end{frame}
% ------------------------------------------------------------------------------
\begin{frame}
  \frametitle{Test configurations}
    \large
    \begin{itemize}
    \item \say{compiler \& clang-tidy}
    \begin{itemize}
    \normalsize
    \item \inlinecode{export CCACHE\_DISABLE=1}
    \item \inlinecode{export CTCACHE\_DISABLE=1}
    \end{itemize}
    \item \say{ccache \& clang-tidy}
    \begin{itemize}
    \normalsize
    \item \inlinecode{unset  CCACHE\_DISABLE}
    \item \inlinecode{export CTCACHE\_DISABLE=1}
    \end{itemize}
    \item \say{compiler \& ctcache}
    \begin{itemize}
    \normalsize
    \item \inlinecode{export CCACHE\_DISABLE=1}
    \item \inlinecode{unset  CTCACHE\_DISABLE}
    \end{itemize}
    \item \say{ccache \& ctcache}
    \begin{itemize}
    \normalsize
    \item \inlinecode{unset CCACHE\_DISABLE}
    \item \inlinecode{unset CTCACHE\_DISABLE}
    \end{itemize}
    \end{itemize}
\end{frame}
% ------------------------------------------------------------------------------
\begin{frame}
  \frametitle{Test execution}
    \Large
    \begin{itemize}
    \item For each test project:
    \begin{itemize}
    \item \inlinecode{cd /path/to/build/dir}
    \item \inlinecode{rm -rf ./}
    \item \inlinecode{cmake ... /path/to/project/source}
    \item do initial build with caches enabled (no measurements),
    \item for each test configuration:
    \begin{itemize}
    \item setup environment variables,
    \item \inlinecode{make clean}
    \item \inlinecode{time make -j N ...}
    \end{itemize}
    \end{itemize}
    \end{itemize}
\end{frame}
% ------------------------------------------------------------------------------
\begin{frame}
  \frametitle{Project 1 -- clean build times {\smaller (compilation caused some swapping on disk)}}
  \fitfig{project1_build_times}
\end{frame}
% ------------------------------------------------------------------------------
\begin{frame}
  \frametitle{Project 2 -- clean build times}
  \fitfig{project2_build_times}
\end{frame}
% ------------------------------------------------------------------------------
\begin{frame}
  \frametitle{Project 3 -- clean build times}
  \fitfig{project3_build_times}
\end{frame}
% ------------------------------------------------------------------------------
\begin{frame}
  \frametitle{Clean build speedups -- $t_{uncached} / t_{cached}$}
  \fitfig{build_speedups}
\end{frame}
% - Conclusion -----------------------------------------------------------------
\section{Conclusion}
% ------------------------------------------------------------------------------
\begin{frame}
  \Huge
  \centering{Conclusions}
\end{frame}
% ------------------------------------------------------------------------------
\begin{frame}
  \frametitle{Conclusions}
    \Large
    \begin{itemize}
    \item Having static analysis (almost) always on is useful during development.
    \item Unnecessary delays from re-checking of unchanged code can be avoided
        by caching.
    \item Using \cmdname{clang-tidy-cache} (and \cmdname{ccache})
        significantly improves build times\footnote{especially clean rebuild
            times} for projects of various sizes.
    \item The achieved speedups\footnote{when using both \cmdname{ccache}
        and \cmdname{ctcache}} are in the range of $\approx$ 5x -- 15x.
    \end{itemize}
\end{frame}
% ------------------------------------------------------------------------------
\begin{frame}
  \frametitle{Overview}
    \LARGE
    \begin{itemize}
    \item \cmdname{clang-tidy-cache} (client \& server)
    \begin{itemize}
    \item are reasonably simple to setup,
    \item provide many configuration options,
    \item provide several deployment options,
	\item can be shared among multiple users\footnote{see extras},
    \item can be integrated into CI pipelines\footnote{Jenkins, Travis, etc.},
    \item work well together with \cmdname{ccache}.
    \end{itemize}
    \end{itemize}
\end{frame}
% ------------------------------------------------------------------------------
\begin{frame}
  \frametitle{Personal experience}
    \Large
    \begin{itemize}
    \item Did development with \cmdname{clang-tidy} checks enabled on
        \say{Project 2} and \say{Project 3} for more than a year.
    \item This helps to find and fix many bugs early.
    \item Also helps to enforce coding guidelines.
    \item With \cmdname{clang-tidy-cache} the build times are kept
        to very reasonable levels.
    \item Build times increase only in rare cases when one of the \say{core}
        headers are changed.
    \end{itemize}
\end{frame}
% ------------------------------------------------------------------------------
\begin{frame}
  \frametitle{Possible future improvements}
    \LARGE
    \begin{itemize}
    \item Support other command-line static analysis tools.
    \item Support for {\em HTTPS} in the server.
    \item Request count over time statistics chart.
    \item Additional command-line options in the client:
    \begin{itemize}
    \item Display server cache statistics on the command-line.
    \item Reset server cache from the command-line.
    \end{itemize}
    \item \ldots
    \end{itemize}
\end{frame}
% ------------------------------------------------------------------------------
\begin{frame}
  \centering
  \Huge
  Thank you!\\Questions?\\
  \vfill
  \Large
  \url{https://github.com/matus-chochlik/ctcache}\\
  \vfill
  \large
  \url{https://github.com/matus-chochlik/ctcache}/doc/overview.pdf
\end{frame}
% - Extras ---------------------------------------------------------------------
\section{Extras}
% ------------------------------------------------------------------------------
\begin{frame}
  \Huge
  \centering{Extras}
\end{frame}
% ------------------------------------------------------------------------------
\begin{frame}
  \frametitle{Sharing cache server among multiple users}
  \Large
  \begin{itemize}
  \item If multiple developers are building the same set of sources\footnote{
      they are working on the same project},
  \item and they have reasonably similar development environments\footnote{
      same platform, same compiler, STL and library versions},
  \item they have access to the same \cmdname{clang-tidy-cache-server} instance,
  \item and they setup the \inlinecode{CTCACHE\_STRIP} variable properly,
  \item they can share each others, cached static analysis results.
  \end{itemize}
\end{frame}
% ------------------------------------------------------------------------------
\begin{frame}
  \frametitle{What prevents analysis result sharing}
  \Large
  \begin{itemize}
  \item The cache works by hashing input strings -- command-line arguments,
      pre-processed source file lines, configuration file lines, etc.
  \item The goal is to get the {\larger same hash} for the {larger \say{same} check}.
  \item Between users with similar environments, the analysis inputs typically
      differ only in {\larger username-dependent} sub-strings\footnote{
      i.e. paths in the \inlinecode{-L}, \inlinecode{-I}, etc. compiler options}.
  \end{itemize}
\end{frame}
% ------------------------------------------------------------------------------
\begin{frame}
  \frametitle{The \inlinecode{CTCACHE\_STRIP} variable}
  \Large
  \begin{itemize}
  \item \inlinecode{CTCACHE\_STRIP} -- List of colon-separated strings, which
      are removed from the input strings.
  \begin{itemize}
  \item For example \inlinecode{CTCACHE\_STRIP="/home/user/myproject:/opt/custom/libs"}.
  \end{itemize}
  \item If the \say{right} strings are stripped by every user, same hashes are
      generated.
  \item {\larger May be somewhat tricky to setup properly.}
  \item {\larger If set-up incorrectly may lead to false positives!}
  \end{itemize}
\end{frame}
% ------------------------------------------------------------------------------
\end{document}
